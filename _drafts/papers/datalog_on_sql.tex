

\title{datakanren}


\author{Philip Zucker}


\abstract{
    Top Down Embedded Relational Programming is well explored in the form of Minikanren, a relative of Prolog.
    Bottom up programming, as exemplified by Datalog, is less explored from the Kanren lens.

}

\section{Introduction}

Relational and Logic programming model computation as the construction of proofs.
Proofs can be thought about from a goal oriented / backwards / top-down / verificationist or an theorem oriented / forward / bottom-up / generating / pragmatist perspective.
In a similar way, a grammar can be thought of in terms of it's parsing problem or as a language generator.

https://www.cs.uoregon.edu/research/summerschool/summer22/lectures/Pfenning1notes.pdf


\section{Simple DataKanren}

Kanren's consist of two ingredients, unification and search.
The essential point of Kanren is to model goals as nondeterminstic state functions.
$ State \rightarrow [State] $

The state is an environment mapping variables to values or other variables. It is a substitution.

Datalog is a simpler system. It uses pattern matching rather than unification.


\cite{datafrog}



\section{Datalog on SQL}

We all live on a substrate of technology produced by industrial forces 
and a basically unimaginable number of smart and hardworking people.

Kanren, as an embedded approach to language design,
leverages the hard work done on Programming language implementations for Scheme, Haskell, 
etc. This makes it possible to get a powerful system for relatively little code compared
to a bespoke from scratch Prolog-like implementation. 

SQL is in essence the most popular largely declarative programming languages
with easily available highly engineered implementations avaialable in all programming languages
for free.

\cite{hytradboi}
\cite{Kiselyov}









% compare speed of souffle and sqlite datalog.

\section{Related Work}

Push datalog is a variant closer in spririt in some respects to Kanren in that it uses the host function call mechanism



% Embedded ASP
%

\cite{https://www.philipzucker.com/notes/Languages/datalog/}

\cite{oatlog}
\cite{yihong egglite}